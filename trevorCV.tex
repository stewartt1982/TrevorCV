% LaTeX file for resume 
% This file uses the resume document class (res.cls)

\documentclass[margin]{res} 

% the margin option causes section titles to appear to the left of body text 
\textwidth=5.2in % increase textwidth to get smaller right margin
%\usepackage{helvetica} % uses helvetica postscript font (download helvetica.sty)
%\usepackage{newcent}   % uses new century schoolbook postscript font 
%\usepackage{html}

\begin{document} 
 
\name{Trevor P. Stewart\\[12pt]} % the \\[12pt] adds a blank line after name
 
\address{{\bf Contact Information} \\ Department of Physics \\ University of Toronto \\ 
60 St. George St. \\ M5S 1A7\\ Toronto, Ontario \\ Canada }

\address{\textit{Office Phone} +1 (416) 978-8478 \\
\textit{Home Phone} +1 (647) 273-8702\\ \textit{E-mail} stewartt@physics.utoronto.ca}

 
\begin{resume} 

\section{RESEARCH INTERESTS}

My research has chiefly focused on high-$Q^2$ neutral and charged current deep inelastic scattering and charm production.  However, I am interested in a wide range of topics in high energy experimental particle physics, especially those that can be studied at the LHC: Higgs and new physics beyond the standard model (BSM) searches and precision QCD and electroweak measurements.  

At ZEUS, my main experience with detector hardware has been the calorimeter and the trigger.  While both of these are of great interest to me, I would also like to gain experience with the operation of other detector components such as muon or tracking detectors.

\section{EDUCATION}

{\bf University of Toronto}, Toronto, Ontario, Canada \\
PhD, Physics \hfill July 2012
\begin{itemize} \itemsep -2pt  % reduce space between items
\item Thesis topic: \textit{Measurement of High-$Q^2$ Neutral Current cross-sections with longitudinally polarised positrons with the ZEUS Detector}
\item Supervisor: Prof. John Martin
\item Area of Study: Experimental Particle Physics
\end{itemize}

MSc, Physics\hfill August 2007 
\begin{itemize} \itemsep -2pt  % reduce space between items
\item Thesis topic: \textit{Charm production in High-$Q^2$ Charged Current Deep Inelastic Scattering}
\item Supervisor: Prof. John Martin
\item Area of Study: Experimental Particle Physics
\end{itemize}

{\bf University of New Brunswick (UNB)}, Fredericton, New Brunswick, Canada \\
I was enrolled in a joint program for science and computer science and received two separate undergraduate degrees.\\
BSc, Physics\hfill December 2005
\begin{itemize} \itemsep -2pt  % reduce space between items 
\item Graduated with Honours 
\end{itemize}

BCS, Computer Science \hfill December 2005
\begin{itemize} \itemsep -2pt  % reduce space between items 
\item Graduated with First Division 
\end{itemize}

\section{CONFERENCES AND PROCEEDINGS}

Trevor Stewart, speaker at The Europhysics Conference of High-Energy Physics, Grenoble, France, July 21-27, 2011. \textit{Measurement of High-$Q^2$ Charged and Neutral Current Deep Inelastic $e^{+}p$ Scattering Cross Sections with a Longitudinally Polarised Positron Beam at HERA.} Proceedings in PoS EPS-HEP2011:367.

Trevor Stewart, speaker at The XIX International Workshop on Deep-Inelastic Scattering and Related Subjects, Newport News, VA, USA, April 11-15, 2011. \textit{Measurement of High-$Q^2$ Neutral and Charged Current Deep Inelastic $e^+p$ Scattering Cross Sections with a Longitudinally Polarised Positron Beam at HERA.} Proceedings to 
appear in \textit{Proceedings of 19th International Workshop on Deep-Inelastic Scattering and Related Subjects.}  

\section{RECENT ZEUS PLENARY TALKS}

Trevor Stewart, speaker at the ZEUS Collaboration Meeting, Hamburg, Germany, September 21-23 2011. \textit{High-$Q^2$ $e^{+}p$ Neutral Current Analysis - Update.}  

Trevor Stewart, speaker at the ZEUS Collaboration Meeting, Hamburg, Germany, October 11-13 2010.  \textit{NC DIS in 06-07 $e^{+}p$ data.} 

Trevor Stewart, speaker at the ZEUS Collaboration Meeting, Kiev, Ukraine, October 5-9 2009. \textit{Charm Production in CC.}   

\section{AWARDS}

\textbf{Major Awards}\\
\begin{itemize}
\item Ontario Graduate Scholarship (OGS), 2007-2008
\item Natural Sciences and Engineering Research Council of Canada (NSERC) Undergraduate Student Research Award, Summer 2004
\end{itemize}
\textbf{University of New Brunswick (UNB)}\\
\begin{itemize}
\item Dr. A. Wilmer Duff Memorial Prize, 2005-2006
\item Frank and  Isa  Pridham Memorial Scholarship, 2001-2002
\item UNB Fredericton Scholarship Guarantee, 2000-2001
\end{itemize}

\section{RESEARCH EXPERIENCE}

Graduate Student \hfill March 2006 to present\\

\begin{itemize}
\item Developed an analysis for charm production in charged current deep inelastic scattering analysis using reconstructed secondary verticies and impact parameter techniques.  This analysis was the first observation of charm production in charged current deep inelastic scattering at ZEUS by this technique.
\item Developed an analysis for high-$Q^2$ neutral current $e^{+}p $ deep inelastic scattering using the HERA polarised positron beam.  This analysis measured the single differential cross-sections, the reduced cross-sections, observed parity violation at high-$Q^2$ and in combination with the previously published HERA-II $e^{-}p$ analysis, made the most precise determination of $xF_3^{\gamma Z}$.  
\item Measured the shape of the $Z_{vtx}$ distribution for the 2006-2007 positron running period at ZEUS.  Showed the need for multiple $Z_{vtx}$ measurements within the running period.  Implemented a correction based on this measurement.
\item Worked on validating the common ntuple (CN) project at ZEUS  using both the charm production in charged current analysis and the neutral current analysis.  The goal of the CN project is to provide ntuples suitable for a wide range of analysis and long term storage of ZEUS data.
\item Contributed to the running, maintenance and calibration of the ZEUS calorimeter. 

\end{itemize}

Summer Research Assistant \hfill May 2005 to August 2005\\

\begin{itemize}
  \item Continuation of work on the ZEUS Third Level Trigger.
  \item Studied the effects of backsplash from the ZEUS calorimeter on the measurement of the kinematic variables using jets.  Implemented this correction for use in high-$Q^2$ neutral current deep inelastic scattering.
\end{itemize}

Summer Research Assistant \hfill May 2004 to August 2004 \\

\begin{itemize}
\item Assisted with the maintenance, operation, debugging and updating of the ZEUS Third Level Trigger (TLT). 
\end{itemize}

\section{TEACHING EXPERIENCE}

{\bf University of Toronto} \\
Teaching Assistant \hfill September 2008 to April 2010 
\begin{itemize}
\item Instructor for PHY 224 - Practical  Physics I (Laboratory)
  \begin{itemize}
    \item 4 semesters
    \item Responsible for the supervision of a 3 hour lab where first year physics and engineering science students are introduced to basic laboratory and data analysis skills through the study of historical experiments from a variety of fields in physics.  Special attention was paid to error analysis, data collection techniques and experimental design.
    \item Responsible for evaluation of weekly laboratory reports, formal experiment write-ups and oral presentations.
  \end{itemize}
\end{itemize}

Teaching Assistant \hfill September 2007 to April 2008 
\begin{itemize}
\item Instructor for PHY 151/152 - Foundations of Physics  (Laboratory)
  \begin{itemize}
    \item 2 semesters
    \item Responsible for the supervision of a 3 hour lab where first year physics students are introduced to the fundamentals of mechanics using micro-computer based data collection and analysis.
    \item Responsible for evaluation of weekly laboratory reports and formal experiment write-ups.
  \end{itemize}
\end{itemize}


{\bf University of New Brunswick} \\
Teaching Assistant \hfill September 2005 to December 2005 
\begin{itemize}
\item Instructor for PHYS 1081 - Foundations of Physics for Engineers (Laboratory)
  \begin{itemize}
    \item 1 semester
    \item Responsible for the supervision of a 3 hour lab where first year engineering students are introduced to the fundamentals of mechanics through hands-on laboratory work.
    \item Responsible for evaluation of weekly laboratory reports and formal experiment write-ups.
  \end{itemize}
\end{itemize}

\section{PUBLICATIONS}

48 ZEUS publications, see \\http://inspirehep.net/search?ln=en\&p=\\FIND+A+STEWART\%2C+TREVOR+P\&of=hb\&action\_search=Search\\
for reference.\\
Participlated in the analysis of the following papers:
\begin{itemize}
\item  S. Chekanov {\it et al.}  [ZEUS Collaboration],\textit{Measurement of high-$Q^2$ neutral current deep inelastic $e^- p$ scattering cross sections with a longitudinally polarised electron beam at HERA} Eur. Phys. J. C {\bf 62}, 625 (2009)
\end{itemize}

Two papers are being readied for publication submission:
\begin{itemize}
\item \textit{Measurement of positron-proton neutral current cross sections at high Bjorken-x with the ZEUS detector at HERA.}
\item \textit{Measurement of high-$Q^2$ neutral current deep inelastic $e^+ p$ scattering cross sections with a longitudinally polarised electron beam at HERA}
  
\end{itemize}

\section{COMPUTER SKILLS}

\textbf{Operating systems:} Linux, Unix, Mac OS X and Windows.\\
\textbf{Programming languages:} Good knowledge of C, C++, FORTRAN, perl and UNIX shell scripting.  Some knowledge of 8086 and 68HC11 assembler and JAVA.\\
\textbf{High energy physics programs:} Batch computing, PAW/HBOOK, ROOT.  Some familiarity with event generators and simulations (GEANT).

\section{LANGUAGES}

Fluent in English and basic knowledge of French and Japanese.

\section{REFERENCES}
\begin{itemize}
  \item Prof. John Martin (martin@physics.utoronto.ca), University of Toronto,\\ Supervisor. 
  \item Prof. Sampa Bhadra (bhadra@yorku.ca), York University.
  \item Prof. Aharon Levy (levy@alzt.tau.ac.il), Tel Aviv University and \\ZEUS Spokesperson.
\end{itemize}
\end{resume} 
\end{document} 



