% LaTeX file for resume 
% This file uses the resume document class (res.cls)

\documentclass[margin]{res} 

% the margin option causes section titles to appear to the left of body text 
\textwidth=5.2in % increase textwidth to get smaller right margin
%\usepackage{helvetica} % uses helvetica postscript font (download helvetica.sty)
%\usepackage{newcent}   % uses new century schoolbook postscript font 
%\usepackage{html}

\begin{document} 
 
\name{Academic CV - Trevor P. Stewart\\[12pt]} % the \\[12pt] adds a blank line after name

\address{{\bf Contact Information} \\ 90 Charing Cres. \\ Fredericton, New Brunswick \\ E3B 4R7 \\ Canada }

\address{\textit{Phone} +15064721173 \\ \textit{Skype-in} +81(0)5031368145 \\ \textit{E-mail} stewartt1982@gmail.com}

 
\begin{resume} 

\section{WORK HISTORY - RESEARCH}
Postdoctoral Research Associate\\/Rutherford Appleton Laboratory \hfill June 2013 to January 2018\\
{\bf T2K}
\begin{itemize} \itemsep -2pt
\item Highly involved in the operation and maintenance of the T2K data acquisition system hardware and software and T2K run coordination.
\item Developed a selection for charged current anti-neutrino $\pi^{-}$ events in an anti-neutrino beam at T2K.
\item In charge of data distribution on the GRID and maintaining of data distribution software.
\end{itemize}

{\bf Hyper-Kamiokande}
\begin{itemize} \itemsep -2pt
\item Research into the requirements needed for the data acquisition system for the planned Hyper-Kamiokande detector.
\item Development of trigger algorithms for the Hyper-Kamiokande detector, concen-trating on triggers to lower the low energy threshold of the detector.
\item Software development work on the Hyper-Kamiokande detector simulation to allow more flexible implementation of digitisation of analogue signals and trigger systems.
\end{itemize}

Graduate Student/University of Toronto \hfill March 2006 to August 2012\\
{\bf ZEUS}
\begin{itemize} \itemsep -2pt
\item Developed an analysis for charm production in charged current deep inelastic scattering analysis using reconstructed secondary vertices and impact parameter techniques. This analysis was the first observation of charm production in charged current deep inelastic scattering at ZEUS by this technique.
\item Developed an analysis for high-$Q^2$ neutral current $e^+ p$ deep inelastic scattering using the HERA polarised positron beam (see publications).
\item Performed studies evaluating the effectiveness of a neural network based electron finder in the context of high-$Q^2$ neutral current $e^+ p$ deep inelastic scattering.
\item Corrections to the ZEUS experiment Monte Carlo detector simulation using data-driven techniques.
\item Worked on validating the common ntuple (CN) project at ZEUS for long term data storage.
\item Contributed to the running, maintenance and calibration of the ZEUS calorimeter and high level Third Level Trigger.
\end{itemize}

Summer Research Assistant/University of Toronto \hfill May 2005 to August 2005\\
{\bf ZEUS}
\begin{itemize} \itemsep -2pt
\item Worked on the ZEUS Third Level Trigger, providing day to day support and testing of new trigger configurations.
\item Studied the effects of backsplash from the ZEUS calorimeter on the measurement of the kinematic variables.
\end{itemize}

Summer Research Assistant/University of Toronto \hfill May 2004 to August 2004\\
{\bf ZEUS}
\begin{itemize} \itemsep -2pt
\item Worked on the ZEUS Third Level Trigger, providing day to day support and testing of new trigger configurations.
\item Studied the reconstruction of kaons in the ZEUS detector.
\end{itemize}

\section{WORK HISTORY - TEACHING}
{\bf University of Toronto}\\
Teaching Assistant \hfill September 2008 to April 2010
\begin{itemize} \itemsep -2pt
\item Instructor for PHY 224 - Practical Physics I (Laboratory)
\item Responsible for the supervision and evaluation of 1st year physics and engineering science students. Special attention paid to error analysis, data collection techniques and experimental design.
\end{itemize}

Teaching Assistant \hfill September 2007 to April 2008
\begin{itemize} \itemsep -2pt
\item Instructor for PHY 151/152 - Foundations of Physics (Laboratory)
\item Responsible for the supervision and evaluation of 1st year physics students. Special attention paid to error analysis, data collection techniques and experimental design.
\end{itemize}

{\bf University of New Brunswick}\\
Teaching Assistant \hfill September 2005 to December 2005
\begin{itemize} \itemsep -2pt
\item Instructor for PHYS 1081 - Foundations of Physics for Engineers (Laboratory)
\item Responsible for the supervision and evaluation of engineering students taking their first physics laboratory course.physics students.
\end{itemize}

\section{SKILLS}
\begin{itemize} \itemsep -2pt
\item Expertise in a number of computer languages: C/C++, FORTRAN, perl, and
python. Some experience with other languages such as assembly languages (x86,
68HC11), R and Java.
\item Extensive experience with Linux and Unix systems for software development. Operational experience with Windows and Mac OSX.
\item Excellent data analysis skills through physics analysis on a variety of experiments.
\item Extensive experience with distributed and GRID computing.
\item Experience with CUDA through the implementation of a prototype Hyper-Kamiokande trigger, and via participation in the Brookhaven GPU Hackathon 2017.
\item Knowledge of commonly used libraries for machine learning such as numpy, scipy, scikit-learn, tensorflow
\item Experienced at giving presentations on technical topics to a wide variety of audiences (both experts and non-experts)
\item Proven ability to manage a team of researchers while acting as T2K run coordinator.
\end{itemize}


\section{EDUCATION}

{\bf University of Toronto}, Toronto, Ontario, Canada \\
PhD, Physics \hfill July 2012
\begin{itemize} \itemsep -2pt  % reduce space between items
\item Thesis topic: \textit{Measurement of High-$Q^2$ Neutral Current cross-sections with longitudinally polarised positrons with the ZEUS Detector}
\end{itemize}

MSc, Physics\hfill August 2007 
\begin{itemize} \itemsep -2pt  % reduce space between items
\item Thesis topic: \textit{Charm production in High-$Q^2$ Charged Current Deep Inelastic Scattering}
\end{itemize}

{\bf University of New Brunswick (UNB)}, Fredericton, New Brunswick, Canada \\

\begin{itemize} \itemsep -2pt  % reduce space between items 
\item BSc, Physics with honours \hfill December 2005
\item BCS, Computer Science First Division \hfill December 2005
\end{itemize}

\section{AWARDS}

\textbf{Awards and Scholarships}\\
\begin{itemize} \itemsep -2pt
\item Ontario Graduate Scholarship (OGS), 2007-2008
\item Natural Sciences and Engineering Research Council of Canada (NSERC) Undergraduate Student Research Award, 2004
\item Dr. A. Wilmer Duff Memorial Prize, 2005-2006
\item Frank and  Isa  Pridham Memorial Scholarship, 2001-2002
\item UNB Fredericton Scholarship Guarantee, 2000-2001
\end{itemize}


\section{PUBLICATIONS}

To see all publications, see\\
http://inspirehep.net/author/profile/T.P.Stewart.1 for reference.\\
Participated in the analysis/preparation of the following papers:
\begin{itemize} \itemsep -2pt
\item S. Chekanov \textit{et al. [ZEUS Collaboration], Measurement of high-$Q^2$ neutral current deep inelastic $e^− p$ scattering cross sections with a longitudinally polarised electron beam at HERA}. Eur. Phys. J. C 62, 625 (2009)
\item H. Abramowicz \textit{et al. [ZEUS Collaboration],Measurement of positron-proton neutral current cross sections at high Bjorken-x with the ZEUS detector at HERA}. Phys. Rev. D89, 072007 (2014)
\item K. Abe \textit{et al. [Hyper-Kamiokande Proto-Collaboration], Hyper-Kamiokande Design Report}. KEK Preprint 2016-21/ICRR-Report-701-2016-1
\end{itemize}

Primary author on:
\begin{itemize} \itemsep -2pt
\item H. Abramowicz \textit{et al. [ZEUS Collaboration],Measurement of high-$Q^2$ neutral current deep inelastic $e^+ p$ scattering cross sections with a longitudinally polarised electron beam at HERA}. Phys.Rev. D87, 052014 (2013)
\item T. Stewart (on behalf of the ZEUS Collaboration) \textit{Measurement of High-$Q^2$ Charged and Neutral Current Deep Inelastic $e^{+}p$ Scattering Cross Sections with a Longitudinally Polarised Positron Beam at HERA}. PoS EPS-HEP2011:367.
\end{itemize}


\section{CONFERENCES, WORKSHOPS AND PROCEEDINGS}

Trevor Stewart, speaker at The Europhysics Conference of High-Energy Physics, Grenoble, France, July 21-27, 2011. \textit{Measurement of High-$Q^2$ Charged and Neutral Current Deep Inelastic $e^{+}p$ Scattering Cross Sections with a Longitudinally Polarised Positron Beam at HERA.} Proceedings in PoS EPS-HEP2011:367.

Trevor Stewart, speaker at The XIX International Workshop on Deep-Inelastic Scattering and Related Subjects, Newport News, VA, USA, April 11-15, 2011. \textit{Measurement of High-$Q^2$ Neutral and Charged Current Deep Inelastic $e^+p$ Scattering Cross Sections with a Longitudinally Polarised Positron Beam at HERA.}

Trevor Stewart, speaker at the $5^{th}$ International Conference on New Frontiers in Physics, Crete, Greece, July 6-14, 2016. \textit{Overview of neutrino physics: Neutrino oscillation measurements and future prospects.}

Trevor Stewart, GPU Hackathon 2017, Brookhaven National Laboratory, 5-9 June
2017. Team “The fasted trigger in the east” developed a fast vertexing trigger for the Hyper-Kamiokande detector, already ported to CUDA, for optimisation. Result was a 5.5x speed gain.




\section{LANGUAGES}

Fluent in English and basic knowledge of French and Japanese.

\section{REFERENCES}
\begin{itemize} \itemsep -2pt
\item Prof. Alfons Weber (alfons.weber@stfc.ac.uk), University of Oxford/STFC,
Line Manager.
\item Prof. Giles Barr (Giles.Barr@physics.ox.ac.uk), University of Oxford.
\item Dr. Helen O’Keeffe (h.okeeffe@lancaster.ac.uk), Lancaster University.
\end{itemize}
\end{resume} 
\end{document} 



