% LaTeX file for resume 
% This file uses the resume document class (res.cls)

%\documentclass[margin]{res} 
\documentclass[margin]{res}
\usepackage{xeCJK} %for Japanese characters
\setCJKmainfont{IPAMincho} % for \rmfamily
\setCJKsansfont{MS Gothic} % for \sffamily

% the margin option causes section titles to appear to the left of body text 
%\textwidth=5.2in % increase textwidth to get smaller right margin
%\usepackage{helvetica} % uses helvetica postscript font (download helvetica.sty)
%\usepackage{newcent}   % uses new century schoolbook postscript font 
%\usepackage{html}

\begin{document} 
 
\name{Trevor P. Stewart\\[12pt]} % the \\[12pt] adds a blank line after name
 
\address{{\bf Contact Information} \\ Until Dec. 26th 2017 \\ 〒319-1111 \\ 茨城県那珂郡東海村舟石川339ー1C \\ パークサイド・リラA201 \\ \\
After Dec. 26th 2017 \\ 90 Charing Cres. \\ Fredericton, New Brunswick \\ E3B 4R7 \\ Canada}


\address{\textit{UK Cellphone} +44(0)7984678145 \\
\textit{Canada} +1(506)4721173\\ \textit{Japanese Cellphone} +81(0)368635429 \\ \\ \textit{E-mail} stewartt1982@gmail.com}

 
\begin{resume} 

\section{PROFESSIONAL SUMMARY}
Experienced researcher in experimental particle physics with experience working on a number of experiments both running and in development.  Have worked on several cross-section analysis on the ZEUS detector and T2K detectors.  On the planned Hyper-Kamiokande detector performed studies on the requirements for a data acquisition and trigger system, and software development for detector simulations.  Strong background in computer science, with experience in a number of different computer languages.  Solid skills in physics analysis, mathematics, statistics and algorithms.  Experience in managing a team of researchers through activities as data acquisition expert and run coordinator on the T2K experiment.  

\section{WORK HISTORY - RESEARCH}
Postdoctoral Research Associate \hfill June 2013 to present\\
\textbf{T2K}
\begin{itemize}
\item Highly involved in the operation and maintenance of the T2K data acquisition system hardware and software and T2K run coordination.
\item In charge of data distribution on the GRID and maintaining of data distribution software
\item Developed a selection for charged current anti-neutrino $\pi^{-}$ events in an anti-neutrino beam at T2K.
\end{itemize}
\textbf{Hyper-Kamiokande}
\begin{itemize}
\item Study requirements needed for a data acquisition system for the planned Hyper-Kamiokande detector.
\item Development of trigger algorithms for the Hyper-Kamiokande detector, concentrating on triggers to lower the low energy threshold of the detector.
\item Software development work on WCSim to allow more flexible implementation of digitisation of analogue signals and trigger systems.
\end{itemize}
Graduate Student \hfill March 2006 to present\\
\textbf{ZEUS}
\begin{itemize}
\item Developed an analysis for charm production in charged current deep inelastic scattering analysis using reconstructed secondary vertices and impact parameter techniques.  This analysis was the first observation of charm production in charged current deep inelastic scattering at ZEUS by this technique.
\item Developed an analysis for high-$Q^2$ neutral current $e^{+}p $ deep inelastic scattering using the HERA polarised positron beam. 
\item Measured the shape of the $Z_{vtx}$ distribution for the 2006-2007 positron running.  Showed the need for more careful treatment of the $Z_{vtx}$ measurement for future analyses.
\item Worked on validating the common ntuple (CN) project at ZEUS for long term data storage.
\item Contributed to the running, maintenance and calibration of the ZEUS calorimeter and high level Third Level Trigger. 
\end{itemize}
Summer Research Assistant \hfill May 2005 to August 2005\\
\begin{itemize}
  \item Worked on the ZEUS Third Level Trigger.
  \item Studied the effects of backsplash from the ZEUS calorimeter on the measurement of the kinematic variables.
\end{itemize}
Summer Research Assistant \hfill May 2004 to August 2004 \\
\begin{itemize}
\item Assisted with the maintenance, operation, debugging and updating of the ZEUS Third Level Trigger. 
\end{itemize}

\section{WORK HISTORY - TEACHING}
{\bf University of Toronto} \\
Teaching Assistant \hfill September 2008 to April 2010 
\begin{itemize}
\item Instructor for PHY 224 - Practical  Physics I (Laboratory)
  \begin{itemize}
    \item Responsible for the supervision and evaluation of first year physics and engineering science students.  Special attention was paid to error analysis, data collection techniques and experimental design.
  \end{itemize}
\end{itemize}
Teaching Assistant \hfill September 2007 to April 2008 
\begin{itemize}
\item Instructor for PHY 151/152 - Foundations of Physics  (Laboratory)
  \begin{itemize}
    \item Responsible for the supervision and evaluation of first year physics students.  Special attention was paid to error analysis, data collection techniques and experimental design.
  \end{itemize}
\end{itemize}
{\bf University of New Brunswick} \\
Teaching Assistant \hfill September 2005 to December 2005 
\begin{itemize}
\item Instructor for PHYS 1081 - Foundations of Physics for Engineers (Laboratory)
  \begin{itemize}
    \item Responsible for the supervision and evaluation of first year engineering students.
  \end{itemize}
\end{itemize}

\section{SKILLS}
\begin{itemize}
  \item Expertise in a number of computer languages: C/C++, FORTRAN, perl, and python.
  \item Some knowledge and experience with assembly languages (x86, 68HC11), CUDA, R and Java
  \item Extensive experience with Linux and Unix systems for software development. Operational experience with Windows and Mac OSX.
  \item Excellent data analysis skills through physics analysis on a variety of experiments. 
  \item Extensive experience with batch and GRID computing.
  \item Proven ability to manage a team of researchers while acting as T2K run coordinator.
\end{itemize}

\section{EDUCATION}

{\bf University of Toronto}, Toronto, Ontario, Canada \\
PhD, Physics \hfill July 2012
\begin{itemize} \itemsep -2pt  % reduce space between items
\item Thesis topic: \textit{Measurement of High-$Q^2$ Neutral Current cross-sections with longitudinally polarised positrons with the ZEUS Detector}
\end{itemize}
MSc, Physics\hfill August 2007 
\begin{itemize} \itemsep -2pt  % reduce space between items
\item Thesis topic: \textit{Charm production in High-$Q^2$ Charged Current Deep Inelastic Scattering}
\end{itemize}
{\bf University of New Brunswick (UNB)}, Fredericton, New Brunswick, Canada \\
BSc, Physics\hfill December 2005
\begin{itemize} \itemsep -2pt  % reduce space between items 
\item Graduated with Honours 
\end{itemize}
BCS, Computer Science \hfill December 2005
\begin{itemize} \itemsep -2pt  % reduce space between items 
\item Graduated with First Division 
\end{itemize}

\section{AWARDS}
\textbf{Major Awards}\\
\begin{itemize}
\item Ontario Graduate Scholarship (OGS), 2007-2008
\item Natural Sciences and Engineering Research Council of Canada (NSERC) Undergraduate Student Research Award, Summer 2004
\end{itemize}
\textbf{University of New Brunswick (UNB)}\\
\begin{itemize}
\item Dr. A. Wilmer Duff Memorial Prize, 2005-2006
\item Frank and  Isa  Pridham Memorial Scholarship, 2001-2002
\item UNB Fredericton Scholarship Guarantee, 2000-2001
\end{itemize}

\section{PUBLICATIONS}
To see all publications, see \\http://inspirehep.net/author/profile/T.P.Stewart.1 
for reference.\\
Participated in the analysis of the following papers:
\begin{itemize}
\item  S. Chekanov {\it et al.}  [ZEUS Collaboration],\textit{Measurement of high-$Q^2$ neutral current deep inelastic $e^- p$ scattering cross sections with a longitudinally polarised electron beam at HERA} Eur. Phys. J. C {\bf 62}, 625 (2009)
\item H. Abramowicz {\it et al.} [ZEUS Collaboration],\textit{Measurement of positron-proton neutral current cross sections at high Bjorken-x with the ZEUS detector at HERA.} Phys. Rev. D89, 072007 (2014)
\end{itemize}
Primary author on:
\begin{itemize}
\item H. Abramowicz {\it et al.} [ZEUS Collaboration],\textit{Measurement of high-$Q^2$ neutral current deep inelastic $e^+ p$ scattering cross sections with a longitudinally polarised electron beam at HERA} Phys.Rev. D87, 052014 (2013)
\end{itemize}
\section{CONFERENCES, WORKSHOPS AND PROCEEDINGS}
Trevor Stewart, speaker at The Europhysics Conference of High-Energy Physics, Grenoble, France, July 21-27, 2011. \textit{Measurement of High-$Q^2$ Charged and Neutral Current Deep Inelastic $e^{+}p$ Scattering Cross Sections with a Longitudinally Polarised Positron Beam at HERA.} Proceedings in PoS EPS-HEP2011:367.

Trevor Stewart, speaker at The XIX International Workshop on Deep-Inelastic Scattering and Related Subjects, Newport News, VA, USA, April 11-15, 2011. \textit{Measurement of High-$Q^2$ Neutral and Charged Current Deep Inelastic $e^+p$ Scattering Cross Sections with a Longitudinally Polarised Positron Beam at HERA.}

Trevor Stewart, speaker at the 5th International Conference on New Frontiers in Physics, Crete, Greece, July 6-14, 2016. \textit{Overview of neutrino physics: Neutrino oscillation measurements and future prospects}

Trevor Stewart, GPU Hackathon 2017, Brookhaven National Laboratory, 5-9 June 2017
Team ``The fasted trigger in the east'' developed a fast vertexing trigger for the Hyper-Kamiokande detector, already ported to CUDA, for optimisation. Result was a 5.5x speed gain. 
\section{LANGUAGES}
Fluent in English and basic knowledge of French and Japanese.

\section{REFERENCES}
\begin{itemize}
  \item Prof. Alfons Weber (alfons.weber@stfc.ac.uk), University of Oxford/STFC,\\ Line Manager. 
  \item Prof. Giles Barr (Giles.Barr@physics.ox.ac.uk), University of Oxford.
  \item Dr. Helen O'Keeffe (h.okeeffe@lancaster.ac.uk), Lancaster University.
\end{itemize}
\end{resume} 
\end{document} 



