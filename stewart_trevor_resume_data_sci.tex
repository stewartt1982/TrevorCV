%________________________________________________________________________________________
% @brief    LaTeX2e Resume for Trevor Stewart
% @author   Trevor Stewart
% @date     March 25 2018
% @info     Based on Latex Resume Template by Chris Paciorek

%________________________________________________________________________________________
\documentclass[margin,line]{resume}
\topmargin -10mm

%\usepackage[utf8]{inputenc}
%\usepackage[plmath,OT4]{polski}

\RequirePackage{color,graphicx}
\usepackage[usenames,dvipsnames]{xcolor}
\usepackage{hyperref}
\definecolor{linkcolour}{rgb}{0,0.2,0.6}
%\definecolor{linkcolour}{rgb}{0,0,0}
\hypersetup{colorlinks,breaklinks,urlcolor=linkcolour, linkcolor=linkcolour}
%\usepackage{fontspec}
%\defaultfontfeatures{Scale=MatchLowercase,Mapping=tex-text} % pozwala na -- i ,,''
%\setromanfont{Gentium}

\newcommand{\superscript}[1]{\ensuremath{^{\textrm{#1}}}}
\newcommand{\thh}[0]{\superscript{th}}
\newcommand{\st}[0]{\superscript{st}}
\newcommand{\nd}[0]{\superscript{nd}}
\newcommand{\rd}[0]{\superscript{rd}}

%\sectionwidth{10}
% resumewidth

\begin{document}
%\title{Curriculum Vitae}
\name{\Large Resume}

\begin{resume}

    %____________________________________________________________________________________
    % Personal Information
    \section{\mysidestyle Personal\\Information}\vspace{2mm}

    \begin{tabular}{@{} l @{\hspace{28mm}} l}
    First name / Surname:    & Trevor Stewart             \\
    E-mail:                  & \href{stewartt1982@gmail.com}{\tt stewartt1982@gmail.com}        \\
    Phone / Skype in:        & +15064721173 / +81(0)5031368145 \\ 
    Profiles: & \href{https://www.linkedin.com/in/trevor-stewart-16153189}{LinkedIn} , \href{https://github.com/stewartt1982}{GitHub}\\
    \end{tabular}
%\vspace{-2mm}

   % ____________________________________________________________________________________
   %  Objective
    \section{\mysidestyle Who am I?}
    Self-motivated PhD-level experimental physicist (ZEUS, T2K and Hyper-Kamiokande experiments) with a background in computer science and a curious, scientific mindset.  Making the transition from academia to private sector data science roles. Extensive experience with advanced mathematics, statistics, problem solving and data analysis, as well as presenting and visualizing complex concepts to diverse audiences.


    %____________________________________________________________________________________
    % Skills
    \section{\mysidestyle Skills}
    \begin{list2}
        \item \textbf{General}: Data analysis, Feature creation and selection, Data cleaning, Computer programming, Advanced Mathematics, Statistics, Quantum field theory, Quantum physics
        \item \textbf{Programming languages}: Python (data science stack: NumPy, SciPy, Pandas, scikit-learn, XGBoost, TensorFlow), C, C++, FORTRAN, Perl, MySQL. Some experience with: x86 and 68HC11 assembly, R, Java, CUDA
        \item \textbf{Technologies}: \LaTeX, Git, SVN, CVS, AWS EC2, MS Office Suite, Linux, Windows, OSX, batch/distributed computing
        \item \textbf{Communication}: Scientific presentations, public science communication, scientific paper writing
        \item \textbf{Team management}: Management of operation of large scientific projects
        \item \textbf{Languages}: English (native), Japanese (basic - conversational)
    \end{list2}

    %\vspace{-2mm}

    %____________________________________________________________________________________
    % Research Interests
    %\section{\mysidestyle Research\\Interests}
    %quantum optics, quantum information, complex systems, mathematical modelling in psychology

%\vspace{3mm}

    %____________________________________________________________________________________
    % Research Experience
    \section{\mysidestyle Experience}

    \begin{list2}

    \item \textbf{Postdoctoral Research Associate at \\\href{https://www.stfc.ac.uk/index.cfm}{Rutherford Appleton Laboratory/STFC}} \hfill{Jun.2013-Jan.2018}\\
      -- Large scale data analysis to extract physics results from terabytes of data, with C++ and Python\\
      -- Developed GPU-accelerated algorithms for selecting physics events in a low signal-to-noise ratio environment\\
      -- Developed data driven methods to improve the accuracy of collected data, specifically time measurements\\
      -- In charge of distributing multi-terabyte datasets to a distributed computing system for use by researchers around the world and ensuring the integrity of the MySQL database\\
      -- Member of a small team of researchers developing requirements for the data acquisition system of a future large scale neutrino oscillation experiment\\
      -- Software development on a detector simulation package to allow for rapid prototyping/testing of analogue signal digitisation, selection of physics signals and simulation of radioactive decays\\
      -- Developed hardware control, automation and monitoring software (with the ability to automatically detect and correct miscommunication errors) for the T2K near detectors\\
      -- Highly involved in both day-to-day and long-term operation, maintenance, and upgrading of a large and complex scientific experiment
    \item \textbf{Graduate Student at \href{https://www.physics.utoronto.ca/}{University of Toronto}} \hfill{Sep.2006-Aug.2012}\\
      -- Large scale data analysis to extract physics results from terabytes of data in C++ and FORTRAN (see Education for details)\\
      -- Developed data driven corrections to monte carlo simulation data to account not modelled real-world effects\\
      -- Development, validation and testing of long-term data storage proposals for multi-terabyte scientific datasets\\
      -- Operation, maintenance and calibration of 2 sub-components of a large particle physics experiment\\
      -- Presented physics research results at several large international conferences
    \item \textbf{Research Assistant \href{https://www.physics.utoronto.ca/}{University of Toronto}} \hfill{Jun. 2004/5-Sep.2004/5 - May. 2006-Sep.2006}\\
      -- Data analysis with the purpose of understanding the reconstruction of kaons using the tracking detectors of a collider experiment\\
      -- Maintenance, and evaluation of new software for selecting interesting events in real-time for a collider experiment\\
      -- Developed a data driven correction to replace an existing theory based correction for improving the accuracy of collected data 
    \item \textbf{Teaching Assistant at \href{https://www.physics.utoronto.ca/}{University of Toronto}} \hfill{Sep.2007-Apr.2008/Sep.2008-Apr.2010}\\
      -- Physics laboratory demonstration, supervision, and evaluation of first year physics and engineering science students
    \item \textbf{Teaching Assistant at \href{http://www.unb.ca/fredericton/science/depts/physics/}{University of New Brunswick}} \hfill{Sep.2005-Dec.2005}\\
      -- Physics laboratory demonstration, supervision, and evaluation of first year engineering students
    \end{list2}
  
  

%\vspace{-2mm}

    %____________________________________________________________________________________
    % Education
    \section{\mysidestyle Education}

    University of Toronto, Toronto, Canada \hfill {2006-2012}\\
    \begin{list2}
      \vspace*{-4mm}
    \item \textbf{PhD Degree}, advisor: John Martin\\
      thesis: \href{https://tspace.library.utoronto.ca/bitstream/1807/34931/1/Trevor_Stewart_P_201211_PhD_thesis.pdf}{Measurement of High-$Q^2$ Neutral Current cross-sections with longitudinally polarised positrons with the ZEUS Detector}
    \item \textbf{MSc Degree}, advisor: John Martin\\
      thesis: Charm production in High-$Q^2$ Charged Current Deep Inelastic Scattering
    \end{list2}
    University of New Brunswick, Fredericton, New Brunswick, Canada \hfill {2000-2005}\\
    \begin{list2}
      \vspace*{-4mm}
    \item \textbf{BSc Degree} Physics, with honours
    \item \textbf{BCS Degree} Computer Science First Division
    \end{list2}

%\vspace{-2mm}


   %____________________________________________________________________________________
    % Awards and Scholarships
    \section{\mysidestyle Awards,\\Scholarships\\}
    \begin{list2}
    \item Ontario Graduate Scholarship (OGS) \hfill{2007}
    \item Natural Sciences and Engineering Research Council of Canada (NSERC)\\ Undergraduate Student Research Award \hfill{2004}
    \item Dr. A. Wilmer Duff Memorial Prize \hfill{2005}
    \item Frank and  Isa  Pridham Memorial Scholarship \hfill{2001}
    \item UNB Fredericton Scholarship Guarantee \hfill {2000}
    \end{list2}

%\vspace{-2mm}
    \section{\mysidestyle Research/Outreach}

    \begin{list2}
    \item {\bf Research papers}: \href{http://inspirehep.net/author/profile/T.P.Stewart.1}{All publications} Primary author: \href{http://inspirehep.net/record/1183813}{1}, \href{https://pos.sissa.it/134/367}{2} Contributor: \href{http://inspirehep.net/record/811152}{1}, \href{http://inspirehep.net/record/1269458}{2}, \href{http://inspirehep.net/record/1501882}{3}
    \item {\bf Major Conferences/Workshops}: \\\href{http://hep2011.insight-outside.fr/}{The Europhysics Conference of High-Energy Physics, Grenoble, France, July 21-27, 2011}\\\href{https://www.jlab.org/conferences/dis2011/}{The XIX International Workshop on Deep-Inelastic Scattering and Related Subjects, Newport News, VA, USA, April 11-15, 2011}\\\href{https://indico.cern.ch/event/442094/}{$5^{th}$ International Conference on New Frontiers in Physics, Crete, Greece, July 6-14, 2016}\\\href{https://www.bnl.gov/gpuhackathon/}{GPU Hackathon 2017, Brookhaven National Laboratory, 5-9 June}
    \item {\bf Public Outreach}: Rutherford Appleton Laboratory Open Day 2015\\Tours of the T2K near detector complex
    \end{list2}
%________________________________________________________________________________________
\end{resume}
\end{document}

%________________________________________________________________________________________
% EOF
