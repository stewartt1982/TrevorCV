\documentclass[12pt,a4paper,oneside]{article} 
\usepackage[margin=1.0in]{geometry}
\usepackage{CJKutf8}

\begin{document}

\title{Cover Letter}
\author{Trevor P. Stewart}
\date{}

\maketitle

\noindent{\bf Contact Information} 
\newline
\newline
\noindent{\textit{Address} \\ 90 Charing Cres.\\Fredericton, New Brunswick\\E3B 4R7\\Canada\\
\newline
\newline  
\noindent{\textit{Phone} +1 (506) 472-1173 \\
\textit{E-mail} stewartt1982@gmail.com}
\newline
\newline

\noindent{To Whom It May Concern:}
\newline
\newline
\noindent{I am writing regarding the JP RPS Engineer positon being offerred on linkedin.com by your company Lam Research.}
\newline
\newline
\noindent{My background is in both computer science (BCS) and physics (BSc, MSc, PhD).  I have spent my professional career to this point in academia, specifically in the field of experimental particle physics.  The main 3 experiments that I have worked on are as follows (details of that work on resume and academic CV):
  \begin{itemize}
  \item ZEUS experiment (DESY, Hamburg, Germany) - An electron-proton collider experiment.  Obtained both an MSc and PhD on data analyses on this experiment.
  \item T2K experiment (Tokaimura, Ibaraki/Hida, Gifu, Japan) - A neutrino oscillation experiment. 
  \item Hyper-Kamiokande (Tokaimura, Ibaraki/Hida, Gifu, Japan) - A proposed successor experiment to the existing Super-Kamiokande experiment.
  \end{itemize}
Due to this background I am proficient in advanced data analysis (using data sets in the multi-TB size range), mathematics, statistics and computing (programming, data structures, algorithms, etc.), and have R\&D experience during the development the data acquisition system for the Hyper-Kamiokande experiment.  Due to the nature of how modern particle physics experiments collect and analyze their data, I am an experienced programmer in C/C++, Perl, Python, FORTRAN and have experience with a number of other languages.  Additionally I have long experience working, upgrading and debugging complex scientific equipment through my roles on the ZEUS and T2K detectors.  On the ZEUS experiment I worked on the the Third Level Trigger (for selecting interesting events from a large background) and the Calorimeter (for measuring the energy and position of particles produced).  In both cases the work was mainly to oversee daily operation of each system, debug problems that occurred during data collection, and provide daily calibrations of the detector subsystems.  As a postdoctoral researcher at the Rutherford Appleton Laboratory I became highly involved in the operation of the T2K detector data acquisition system (DAQ) with roles varying from day-to-day operation, diagnostics and maintenance to upgrading the DAQ and writing backend software.  In my role as T2K Run Coordinator (the person in charge of the overall operation of the T2K experiment during data taking) I was required to obtain a working understanding of not just the DAQ system, but all T2K sub-detectors.
}
\newline
\newline
\noindent{I have recently decided to make the move from academia to private industry, primarily due to a wish for a more permanent position (particularly a position in Japan) rather than moving countries/continents every 2-3 years as a postdoctoral researcher.  Given my background in data analysis, computer science/programming, and scientific mindset, and long experience with complex scientific equipment, I feel that I would be a good fit for this position.
}
\newline
\newline
\noindent{I appreciate your consideration of this application.  I may be contacted using the information supplied above. }
\newline
\newline
Sincerely,
\newline
Trevor Stewart
\newline
\newline



\end{document}
