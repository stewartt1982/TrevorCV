\documentclass[12pt,a4paper,oneside]{article} 
\usepackage[margin=1.0in]{geometry}

\begin{document}

\title{Cover Letter}
\author{Trevor P. Stewart}
\date{}

\maketitle

\noindent{\bf Contact Information} 
\newline
\newline
\noindent{\textit{Address} \\ 90 Charing Cres.\\Fredericton, New Brunswick\\E3B 4R7\\Canada\\
\newline
\newline  
\noindent{\textit{Phone} +1 (506) 472-1173 \\
\textit{E-mail} stewartt1982@gmail.com}
\newline
\newline

\noindent{To Whom it May Concern:}
\newline
\newline

\noindent{I'm writing regarding the position, Service Engineer Bruker Optics Japan, which was recently posted on linkedin.com.}
\newline
\newline
\noindent{My background is in both computer science (BCS), and physics (BSC, MSc, PhD).  The bulk of my professional career to date has been in academia, specifically in the field of experimental particle physics on 3 experiments: ZEUS, T2K and Hyper-Kamiokande.  Due to this background I am an experienced programmer in a variety of languages (Perl, python, C/C++ etc), proficient at data analysis and have given dozens of presentations to audiences with a wide range of technical knowledge.}
\newline
\newline
\noindent{One of the lesser known skills required of experimental particle physicists is an understanding of complex scientific instruments, and electronic equipment.} 
\newline
\newline
\noindent{Modern particle physics experiments are complicated mechanical and electronic devices made up of many independent subsystems which must work together to collect good data for analysis.  This leads to the concept of 'service work' where each person on the experiment has a task(s) to perform in order to properly operate the experiment. Service work can take a number of forms 1) data analysis (such as for low level calibration of the detector), or more relevant to this position, 2) maintaining, debugging, repairing and upgrading the mechanical and electronic portions of the detector.  I have had the opportunity to participate in both forms of service work, but the 2) form has been where I have spent most time and effort.  On the ZEUS experiment I was involved in two systems, the Third Level Trigger (for selecting interesting events from a large background) and the Calorimeter (for measuring the energy and position of particles produced).  In both cases the work was mainly to oversee daily operation of each system, debug problems that occurred during data collection, and provide daily calibrations of the detector subsystems.  As a postdoctoral researcher at the Rutherford Appleton Laboratory I became highly involved in the operation of the T2K detector data acquisition system (DAQ) with roles varying from day-to-day operation, diagnostics and maintenance to upgrading the DAQ and writing backend software.  Additionally in my role as T2K Run Coordinator (the person in charge of the overall operation of the T2K experiment during data taking) I was required to obtain a working understanding of not just the DAQ system, but all T2K sub-detectors.}
\newline
\newline
\noindent{I appreciate your consideration of this application.  I may be contacted using the information supplied above. }
\newline
\newline
Sincerely,
\newline
Trevor Stewart



\end{document}
